\documentclass[11pt]{article}
\usepackage{amsmath, amsfonts, amsthm, amssymb}  % Some math symbols
\usepackage{enumerate}
\usepackage{fullpage}
\usepackage{color}
\usepackage[x11names, rgb]{xcolor}
\usepackage{tikz}
\usepackage{graphicx}
\usepackage{listings}
\usepackage{fancyhdr}

% \usepackage{fontspec}
% \setmainfont{Times New Roman}

\renewcommand*{\familydefault}{\sfdefault}

\setlength{\parindent}{0pt}
\setlength{\parskip}{5pt plus 1pt}
\pagestyle{empty}

\pagestyle{fancy}
\fancypagestyle{firststyle}
{
   \lhead{\myname \\ \myandrew \\ \today \\ \vspace*{-.4em}}
   \rhead{15-221 \\ Fall 2014 \\ Section A \\ \vspace*{-.4em}}
   \setlength{\headsep}{45pt}
}

\newcommand{\myname}{Your name here}
\newcommand{\myandrew}{andrewid@andrew.cmu.edu}
\newcommand{\mytitle}{Project Proposal}
\title{Explainshell Chrome Extension \\ \vspace*{.5em} \Large\mytitle}
\date{}
%%%%%%%%%%%%%%%%%%%%%%%%%%%%%%%%%%%%%%%%%%%%%%%%%%%%%%%%%%%

\begin{document}
\pagenumbering{gobble} 
\clearpage\maketitle
\thispagestyle{firststyle}
\newpage
\pagenumbering{arabic}  
\lhead{\myname}
\rhead{\thepage}
\setlength{\voffset}{-50pt}
\setlength{\headsep}{25pt}

\section{Plan}

We plan to develop a series of technologies that extend the current functionalities of the explainshell project. Our development will include a Chrome extension and a back-end server that provides analytics about trending shell commands. We will also contribute to the current code base of the open source explainshell project. Specifically, our project will be separated into the following four phases. 

\par{\bf Phase 1}\\ 
In Phase 1, we will build a simple Chrome extension that allows user to look up the usage of shell command by highlighting any text in the browser window. Upon highlighting, a button will appear next to the highlighted text region. The user may click the button, which will instantly redirect the user to the corresponding usage documentation at the explainshell website in a new browser tab.

\par{\bf Phase 2}\\
In Phase 2, we will refine our Chrome extension developed in Phase 1. Instead of asking the user to highlight the shell command that they want to look up, the Chrome extension will now automatically parse and search the web page to identify any shell commands. Upon identifying potential commands, the extension will turn the command into a clickable link, which will redirect the user to the usage documentation at the explainshell website.

\par{\bf Phase 3}\\
In Phase 3, we will contribute to the code base of the explainshell project by adding an HTTP API, since this would also help us further improve our chrome extension. With the API, the Chrome extension may now display a pop-up window in the page, whenever the user hovers / clicks on the identified shell commands, whereas previously we always redirect the user to the documentation web page.

\par{\bf Phase 4}\\
In Phase 4, we will develop a back-end server that provides analytics that allow users to view trending searches and referral sites. [Insert more description here]

For our project, we have set up a Git repository that allows us to coordinate and keep track of code changes easily. We have also set up a Google Drive to keep track of our progress. We will set up team meetings at the end of each phase by email or texting.

\end{document}
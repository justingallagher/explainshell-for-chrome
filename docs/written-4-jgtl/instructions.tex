\documentclass[11pt]{article}
\usepackage{amsmath, amsfonts, amsthm, amssymb}  % Some math symbols
\usepackage{enumerate}
\usepackage{fullpage}
\usepackage{color}
\usepackage[x11names, rgb]{xcolor}
\usepackage{tikz}
\usepackage{graphicx}
\usepackage{listings}
\usepackage{fancyhdr}
\usepackage{pdflscape}
\usepackage{hyperref}

% \usepackage{fontspec}
% \setmainfont{Times New Roman}

\renewcommand*{\familydefault}{\sfdefault}

\setlength{\parindent}{0pt}
\setlength{\parskip}{6pt}
\pagestyle{empty}

\pagestyle{fancy}
\fancypagestyle{firststyle}
{
   \lhead{\myname \\ \myandrew \\ \today \\ \vspace*{-.5em}}
   \rhead{15-221 \\ Fall 2014 \\ Section A \\ \vspace*{-.5em}}
   \setlength{\headsep}{50pt}
}

\newcommand{\myname}{Justin Gallagher, Ted Li}
\newcommand{\myandrew}{Group 20}
\newcommand{\mytitle}{Instructions}
\title{Creating a Simple Chrome Extension\\ \vspace*{.5em} \Large\mytitle}
\date{}
%%%%%%%%%%%%%%%%%%%%%%%%%%%%%%%%%%%%%%%%%%%%%%%%%%%%%%%%%%%

\begin{document}

\pagenumbering{gobble} 
\author{~\\
\normalsize {\bf Submitted to}\\
\normalsize Thomas M. Keating\\
\normalsize Assistant Teaching Professor\\
\normalsize School of Computer Science\\
\normalsize Carnegie Mellon University\vspace*{2em}\\
\normalsize {\bf Prepared by}\\
\normalsize Justin Gallagher\\
\normalsize Ted Li\vspace*{2em}\\
\normalsize School of Computer Science\\
\normalsize Carnegie Mellon University\\
\normalsize \today}

\clearpage\maketitle
\thispagestyle{firststyle}

\newpage
\lhead{\myname}
\rhead{\thepage}
\setlength{\headsep}{25pt}
\tableofcontents
\newpage
\pagenumbering{arabic} 
\setlength{\voffset}{-50pt}
\setlength{\headsep}{25pt}

\section{Overview}

\subsection{Introduction}

Chrome extensions allow you to add features and functionality to Google's Chrome browser without modifying source code. This guide will walk you through creating a simple extension and uploading it to the Chrome Web Store for the public to download.

The tutorial has X steps and can be completed in approximately Y minutes.

\subsection{Motivation}

Chrome extensions allow you to add features and functionality to Google's Chrome browser without modifying source code. The framework utilizes web development technologies such as HTML, CSS, and JavaScript which may already be familiar to you. Extensions are also modular, so you can build many which work in their own environments without conflict. Finally, Chrome extensions can be easily distributed to users through the Chrome Web Store, allowing wide adoption of your creations.

\subsection{Requirements}

For this guide, you will need:

\begin{enumerate}
	\item A Internet connected Windows computer with Google Chrome installed.
	\item A text editor in which you are proficient.
\end{enumerate}

\subsection{Goal}

By the end of this guide, you will have created a simple Chrome extension which produces a pop-up window reading "Hello world!" when a button in the top right toolbar is clicked. The extension will also be available for the general public to download onto their browser from the Chrome Web Store.

\subsection{Cautions and Warnings}

An incorrectly written extension can cause undesired behavior resulting in an unstable browser. Uploading such an extension can cause it to be removed from the Chrome Web Store and your developer account to be banned. Remember to test your extension thoroughly before uploading!

\newpage

\section{Adding the Extension to Chrome}

This step outlines how to add the extension you wrote to your installation of Chrome, and how to run the application for yourself.

\begin{enumerate}
	\item Open your Google Chrome browser and visit \texttt{chrome://extensions}.
	\item At the top right of the page, ensure that the ``Developer Mode'' checkbox is checked, as shown in \emph{Figure \ref{fig:devmode}}. If it is not, click the box next to the text label to activate Developer Mode.

	\begin{figure}[htb]
	\centering
	\includegraphics[width=1\textwidth]{figures/devmode.png}
	\caption{Enabling Chrome developer mode\label{fig:devmode}}
	\end{figure}

	\item Press the ``Load unpacked extension...'' button and select the folder containing your \texttt{manifest.json} file, as shown in \emph{Figure \ref{fig:loadext}}. Press ``OK'' to load your extension.

	\begin{figure}[htb]
	\centering
	\includegraphics[height=0.5\textwidth]{figures/loadext.png}
	\caption{Adding your extension to Chrome\label{fig:loadext}}
	\end{figure}

	\item When your extension is properly loaded, you should see an entry for ``My Chrome Extension'' in the extensions list, as shown in \emph{Figure \ref{fig:loadedext}}.

	\begin{figure}[htb]
	\centering
	\includegraphics[width=1\textwidth]{figures/loadedext.png}
	\caption{A successfully loaded extension\label{fig:loadedext}}
	\end{figure}
\end{enumerate}

Proceed to the next step to learn how to upload your extension to the Chrome Web Store for others to download.

\newpage

\section{Uploading the Extension to the Chrome Web Store}

\end{document}

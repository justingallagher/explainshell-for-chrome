\documentclass[11pt]{article}
\usepackage{amsmath, amsfonts, amsthm, amssymb}  % Some math symbols
\usepackage{enumerate}
\usepackage{fullpage}
\usepackage{color}
\usepackage[x11names, rgb]{xcolor}
\usepackage{tikz}
\usepackage{graphicx}
\usepackage{listings}
\usepackage{fancyhdr}
\usepackage{pdflscape}
\usepackage{hyperref}

% \usepackage{fontspec}
% \setmainfont{Times New Roman}

\renewcommand*{\familydefault}{\sfdefault}

\setlength{\parindent}{0pt}
\setlength{\parskip}{6pt}
\pagestyle{empty}

\pagestyle{fancy}
\fancypagestyle{firststyle}
{%
  \lhead{\myname{} \\ \myandrew{} \\ \today \\ \vspace*{-.5em}}
  \rhead{15{-}221 \\ Fall 2014 \\ Section A \\ \vspace*{-.5em}}
  \setlength{\headsep}{50pt}
}

\fancypagestyle{zerostyle}
{%
  \renewcommand{\headrulewidth}{0pt}
}

\newcommand{\myname}{Justin Gallagher, Ted Li, Jacob Zimmerman, Howard Chen}
\newcommand{\myandrew}{Group 20}
\newcommand{\mytitle}{Progress Report}
\title{ExplainShell for Chrome \\ \vspace*{.5em} \Large\mytitle}
\date{}
%%%%%%%%%%%%%%%%%%%%%%%%%%%%%%%%%%%%%%%%%%%%%%%%%%%%%%%%%%%

\begin{document}
\pagenumbering{gobble}
\author{~\\
\normalsize {\bf Submitted to}\\
\normalsize Thomas M. Keating\\
\normalsize Assistant Teaching Professor\\
\normalsize School of Computer Science\\
\normalsize Carnegie Mellon University\vspace*{2em}\\
\normalsize {\bf Prepared by}\\
\normalsize Justin Gallagher\\
\normalsize Ted Li\\
\normalsize Jacob Zimmerman\\
\normalsize Howard Chen\vspace*{2em}\\
\normalsize School of Computer Science\\
\normalsize Carnegie Mellon University\\
\normalsize \today}

\clearpage\maketitle
\thispagestyle{firststyle}

\newpage
\lhead{\myname}
\rhead{\thepage}
\setlength{\headsep}{25pt}
\tableofcontents
\newpage
\pagenumbering{arabic}
\setlength{\voffset}{-50pt}
\setlength{\headsep}{25pt}

\section{Overview}

\subsection{Report}

This report intends to inform the client on our current progress in building
ExplainShell for Chrome. Specifically, we will cover additional information that
we have found relevant to our project, the goals we have accomplished, what we
still need to complete, required changes to our original procedure, and our plan
for finishing the project at the deadline. We will conclude with a discussion of
what exactly we expect to produce when finished.

\subsection{Project}

ExplainShell for Chrome is intended to help the user understand Unix shell
commands found online. By hovering over a command in Google Chrome, a pop-up
window will appear which breaks down the command into its component parts. Each
segment will be thoroughly and explicitly explained through the aid of
explainshell.com, which already catalogues this information. Our application
eliminates the need to copy and paste the command into explainshell - instead,
we can draw a smaller page which displays this information on hover. Thus, the
user does not need to navigate to another page or even stop looking at their
content. This system will both educate users in how to write their own Unix
commands, and allow them to fully understand the commands they run and avoid
running malicious commands.

\section{Literature Review}

\section{Progress}

\subsection{General Assessment}
\subsection{Status}
\subsection{Projections}

\section{Recommendations}

\section{Discussion}

\end{document}

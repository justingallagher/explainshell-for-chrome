\documentclass[11pt]{article}
\usepackage{amsmath, amsfonts, amsthm, amssymb}  % Some math symbols
\usepackage{enumerate}
\usepackage{fullpage}
\usepackage{color}
\usepackage[x11names, rgb]{xcolor}
\usepackage{tikz}
\usepackage{graphicx}
\usepackage{listings}
\usepackage{fancyhdr}
\usepackage{pdflscape}
\usepackage{hyperref}

% \usepackage{fontspec}
% \setmainfont{Times New Roman}

\renewcommand*{\familydefault}{\sfdefault}

\setlength{\parindent}{0pt}
\setlength{\parskip}{6pt}
\pagestyle{empty}

\pagestyle{fancy}
\fancypagestyle{firststyle}
{%
  \lhead{\myname{} \\ \myandrew{} \\ \today \\ \vspace*{-.5em}}
  \rhead{15{-}221 \\ Fall 2014 \\ Section A \\ \vspace*{-.5em}}
  \setlength{\headsep}{50pt}
}

\fancypagestyle{zerostyle}
{%
  \renewcommand{\headrulewidth}{0pt}
}

\newcommand{\myname}{Justin Gallagher, Ted Li, Jacob Zimmerman, Howard Chen}
\newcommand{\myandrew}{Group 20}
\newcommand{\mytitle}{Progress Report}
\title{ExplainShell for Chrome \\ \vspace*{.5em} \Large\mytitle}
\date{}
%%%%%%%%%%%%%%%%%%%%%%%%%%%%%%%%%%%%%%%%%%%%%%%%%%%%%%%%%%%

\begin{document}
\pagenumbering{gobble}
\author{~\\
\normalsize {\bf Submitted to}\\
\normalsize Thomas M. Keating\\
\normalsize Assistant Teaching Professor\\
\normalsize School of Computer Science\\
\normalsize Carnegie Mellon University\vspace*{2em}\\
\normalsize {\bf Prepared by}\\
\normalsize Justin Gallagher\\
\normalsize Ted Li\\
\normalsize Jacob Zimmerman\\
\normalsize Howard Chen\vspace*{2em}\\
\normalsize School of Computer Science\\
\normalsize Carnegie Mellon University\\
\normalsize \today}

\clearpage\maketitle
\thispagestyle{firststyle}

\newpage
\lhead{\myname}
\rhead{\thepage}
\setlength{\headsep}{25pt}
\tableofcontents
\newpage
\pagenumbering{arabic}
\setlength{\voffset}{-50pt}
\setlength{\headsep}{25pt}

\section{Overview}

\subsection{Purpose}

This report intends to inform the client of our current progress in building
ExplainShell for Chrome. Specifically, we will cover additional information that
we have found relevant to our project, the goals we have accomplished, what we
still need to complete, required changes to our original procedure, and our plan
for finishing the project at the deadline. We will conclude with a discussion of
what exactly we expect to produce when finished.

\subsection{Project}

ExplainShell for Chrome is intended to help the user understand Unix shell
commands found online. With the aid of our Chrome extension, users will be able
to click on commands and be directed to the relevant \url{explainshell.com} page.
Our application eliminates the need to copy and paste the command into
\url{explainshell.com} and places a friendly reminder of ExplainShell's utility
in the reader's view. Combined, these two aspects eliminate the largest
barriers to people using \url{explainshell.com} freely, thus promoting its use.

As a complement to this utility, each click that users generate by following
links to \url{explainshell.com} from our Chrome extension will be sent off
anonymously to our ExplainShell Trends site, which will compile a database of
clicks which we will be able to query for statistics such as most recently
clicked commands, top referring sites and pages, most popular commands, and
more. Through this site, ExplainShell for Chrome users \textit{and others} will
be able to learn about new, helpful shell commands.

Thus, this system will both help bring attention to commands that users don't
understand as well as introduce them to new content generated by their peers.
It solves issues related to malicious code execution (because potential victims
will be more wary about running unknown commands), Unix shell education, and
community involvement.

\section{Literature Review}

We have found a number of reference materials useful along the development
process, mostly relating to the actual implementation of Chrome extensions and
web apps in Node.js. Of the sites we've visited, those that we document here
have been the most crucial to our current progress.

\subsection{TODO -- Chrome extension reference}
% TODO: Ted: fill in example doc for the Chrome extension

\subsection{Setting Up a Node.js Stack with Express \& MongoDB}

\url{http://cwbuecheler.com/web/tutorials/2013/node-express-mongo/}

\section{Progress}

\subsection{General Assessment}
\subsection{Status}
\subsection{Projections}

\section{Recommendations}

Drop the popup (or reschedule to after done debugging Phase 2)

\section{Discussion}

\end{document}
